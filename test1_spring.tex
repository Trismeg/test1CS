\documentclass{article}

\usepackage{pgfplots}
\usepackage[margin=0.75in, paperwidth=8.5in, paperheight=11in]{geometry}
\usepackage{setspace}
\usepackage{fancyvrb} % extended verbatim environments
\usepackage{framed}%To get shade behind text

\definecolor{shadecolor}{rgb}{0.9,0.9,0.9}%setting shade color


\begin{document}
\pagenumbering{gobble}

\doublespacing
\textbf{IB Computer Science }                        %%%(class number and section) 
 \hfill                             %%%(date of test)
$ {\bf Name: } \underline{\hspace{2.5in}}$(3 points)

\begin{centering}
\vspace{1cm}
\textbf{Spring: Exam 1}\\
\end{centering}
\vspace{1cm}
 

  
 
 $\bf{1)}$ Write a Java class that prints the following. (10 points)
  
   \begin{verbatim}
 Hola mundo!
 \end{verbatim}
 
 $\bf{2)}$ Write a method named "counter" that takes two integers, A and B, and prints the numbers from A to B. (20 points)
  \vspace{0.5cm}
 
  $\bf{3)}$ Write some Java code that will fill an array with the numbers from 10 to 100.  (20 points)
   \vspace{0.5cm}
   
  $\bf{4)}$ Write a method named "average" that will return the average value of an integer array. \\
   It should return a double.  (20 points)
   \vspace{0.5cm}
   
  $\bf{5)}$ Draw the truth table for OR and XOR.  (10 points)
   \vspace{0.5cm}
   
  $\bf{6)}$ Write a method for the XOR operator named "xor".  It should take two booleans as arguments and return a boolean.  (20 points)
   \vspace{0.5cm}
   
   $\bf{7)}$ Explain how to compile and run a program "hello.java" from the command line.  (10 points)
  


  
  
  
    \newpage
  
  $\bf{3)}$  Draw the output of the following program. (20 points)
  \begin{verbatim}
from graphics import *

win=GraphWin()
win.setCoords(-10,-10,10,10)

for i in range(1,6):
    circ=Circle(Point(i,i),(2**0.5) * i)
    rec=Rectangle(Point(-i,-i),Point(0,0))
    t = Text(Point(-5,9-i), "yo")
    circ.draw(win)
    rec.draw(win)
    t.draw(win)
    
    \end{verbatim}
    
   
  \newpage
  
  $\bf{4)}$ Perform the following conversions. (10 points each)
  \vspace{0.5cm}
  
   a. Convert the Base5 number 432 to Base10. 
  
  b. Convert the binary number 1010001 to Base10. 
  
  c. Convert the Base10 number 532 to Base5.
  
  

 
\end{document}